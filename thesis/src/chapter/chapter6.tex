\documentclass[../Kamil_Kowalewski_Main.tex]{subfiles}

\begin{document} {
    Celem pierwszej części niniejszej pracy magisterskiej było wprowadzanie do tematyki
    dotyczącej zakupów przy użyciu internetowych portali ogłoszeniowych. Został
    przedstawiony fakt niesamowitego rozwoju tej formy sprzedaży na podstawie badań
    rynku z~kilku poprzednich lat. Kolejne sekcje prezentowały niepodważalne zalety
    handlu w~internecie, ale również nowe problemy i~sytuacje, z~którymi muszą się
    mierzyć ich użytkownicy.

    Następne rozdziały prezentowały aktualny stan wiedzy w~literaturze naukowej
    w~obszarze dokonywania rekomendacji ofert, które można określić jako wiarygodne. Na
    podstawie ich analizy zostały przedstawione potencjalne obszary ulepszeń dostępnych
    rozwiązań. W~oparciu o~nie została opracowana autorska metoda oceny wiarygodności
    ofert na portalach ogłoszeniowych w~dwóch wariantach. Została również przedstawiona
    hipoteza badawcza.

    W~kolejnym już czwartym z~kolei, rozdziale zostało przedstawione środowisko
    eksperymentalne, dokładne informacje na temat implementacji autorskiej metody oraz
    wybranej metody z~literatury. Co warto dodać zostały przedstawione
    i~scharakteryzowane zbiory danych wykorzystane w~badaniach. Co więcej, autor
    w~szczegółowy sposób przedstawił sam proces przygotowywania danych i~ekstrakcji
    cech, aby była możliwość wykorzystania ich w~możliwie jak najlepszy sposób.

    Następny rozdział przedstawia wykonane eksperymenty na wybranych zbiorach danych
    oraz uzyskane wyniki. Na ich podstawie można stwierdzić, że postawiona hipoteza
    badawcza okazała się prawdziwa, bo w~przeprowadzonych eksperymentach metoda wykazała
    się wyższą skutecznością niż wybrana metoda z~literatury.

    Tematyka dotycząca systemów rekomendacji nakierowanych na dokonywanie rekomendacji
    w~branży e-commerce z~pewnością w~następnych latach będzie notowała duże przyrosty
    oraz będzie stawała się coraz bardziej popularna, tak samo z~resztą jak same
    portale ogłoszeniowe. Zdaniem autora niniejszej pracy dyplomowej, przedstawiona
    autorska metoda jest kolejny krokiem naprzód w~ulepszaniu wyspecjalizowanych
    systemów rekomendacji. Bardzo możliwe, że w~najbliższych latach nastąpi swojego
    rodzaju przełom i~same portale ogłoszeniowe będą zapewniać więcej informacji, na
    podstawie, których analiza i~rekomendacje będą mogły być jeszcze bardziej
    wyrafinowane i~złożone. Dzięki temu ich skuteczność będzie jeszcze większa a~co za
    tym idzie, wygoda użytkowników takich portali wstąpi na zupełnie inny poziom.
}
\end{document}
