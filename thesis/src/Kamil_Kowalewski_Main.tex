\documentclass[a4paper,12pt,polish,twoside]{extreport}

% ustawienia marginesów
\usepackage{geometry}
\geometry{
    a4paper,
    top=28mm,
    inner=30mm,
    outer=30mm,
    bottom=28mm,
}
% Ustawienie wciecia paragrafow
\usepackage{indentfirst}

% Wylaczenie dzielenia wyrazow gdy miejsce w linii sie konczy
\tolerance=1
\emergencystretch=\maxdimen
\hyphenpenalty=10000
\hbadness=10000

% Ustawnienie spisu tresci aby rowniez zawieral \subsubsection
\setcounter{tocdepth}{4}
\setcounter{secnumdepth}{4}

% Komenda wyrozniajca fragmenty komend terminala w tekscie
\newcommand{\shellcmd}[1]{\texttt{\footnotesize #1}}

% ustawienie interlinii na 1.5
\renewcommand{\baselinestretch}{1.5}

% zmiana punktowania list na myślniki
%\renewcommand\labelitemi{---}

% Pakiet do zmiany odleglosci w itemize
\usepackage{enumitem}

% Pakiet to wielolinijkowych komórek w tabelach
\usepackage{tabularx}
\newcolumntype{L}{>{\raggedright\arraybackslash}X}

%----------------------------------%
% Development with local python script (pdflatex)
\usepackage[utf8]{inputenc}
\usepackage[T1]{fontenc}
\usepackage[polish]{babel}
\usepackage{polski}
\usepackage{times}

% Release with Overleaf, import zipped project and choose Xelatex as compiler in Overleaf
%\usepackage{fontspec}
%\setmainfont{Times New Roman}
%\usepackage[polish]{babel}
%----------------------------------%

% bibliografia
\usepackage{csquotes}
\usepackage[backend=biber,style=numeric,sorting=none]{biblatex}
\addbibresource{bibliography.bib}
\usepackage{hyperref}

% listingi i spis listingów
\usepackage{listings}
\usepackage[center]{caption}
\DeclareCaptionType{code}[Listing][Spis listingów]
\lstset{breaklines=true,basicstyle=\ttfamily\scriptsize,numbers=left}

% rysunki i spis rysunków
\usepackage{float}
\usepackage{graphicx}
\graphicspath{ {./img/} }
\usepackage[nottoc]{tocbibind}

% tabele
\floatstyle{plaintop}
\restylefloat{table}
\captionsetup[table]{name=Tabela}

% strona tytułowa
\usepackage{pdfpages}
\usepackage{subfiles}

% treść
\begin{document} {
    \pagenumbering{gobble}

    \includepdf[pages={1}]{pdf/title_page}
    \null\newpage

    \pagenumbering{arabic}
    \tableofcontents

    \chapter*{Streszczenie}
    \label{chapter0:streszczenie} {
        \subfile{chapter/chapter0.tex}
    }
    \addcontentsline{toc}{chapter}{Streszczenie}

    \chapter{Wstęp}
    \label{chapter1:wstep} {
        \subfile{chapter/chapter1.tex}
    }

    \chapter{Przegląd literatury i wprowadzenie teoretyczne}
    \label{chapter2:przeglad_literatury} {
        \subfile{chapter/chapter2.tex}
    }

    \chapter{Opracowana metoda}
    \label{chapter3:metoda} {
        \subfile{chapter/chapter3.tex}
    }

    \chapter{Środowisko eksperymentalne}
    \label{chapter4:srodowisko_eksperymentalne} {
        \subfile{chapter/chapter4.tex}
    }

    \chapter{Wyniki eksperymentów}
    \label{chapter5:eksperymenty} {
        \subfile{chapter/chapter5.tex}
    }

    \chapter{Podsumowanie i wnioski}
    \label{chapter6:podsumowanie} {
        \subfile{chapter/chapter6.tex}
    }

    \newpage
    \listoffigures

    \newpage
    \listofcodes

    \newpage
    \renewcommand{\listtablename}{Spis tabel}
    \listoftables

    \newpage
    \printbibliography[heading=bibintoc,title={Bibliografia}]
}
\end{document}
