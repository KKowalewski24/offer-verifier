\documentclass[../Kamil_Kowalewski_Main.tex]{subfiles}
\usepackage{subfiles}

\begin{document} {

    W~ramach niniejszej pracy magisterskiej zostały przeprowadzone badania w~formie
    eksperymentów. Celem tego było określenie skuteczności stworzonej metody w~stosunku
    do zaimplementowanej metody z~literatury opisanej w~sekcji
    \ref{chapter4:srodowisko_eksperymentalne:impl_literaturowej_metody}. Do tego celu
    zostały wykorzystane zbiory danych opisane w~sekcji
    \ref{chapter4:srodowisko_eksperymentalne:zbiory_danych}. Dla każdej oferty ze
    wszystkich zbiorów danych została przeprowadzona ręczna oceną wiarygodności oferty
    przez człowieka. Krokami, jakie zostały wykonane w~ramach eksperymentów są:

    \begin{enumerate}[noitemsep,topsep=1pt]
        \item Wybór zbioru danych, w tym przypadku od \#1 do \#4
        \item Przeprowadzanie eksperymentów dla każdego z poniższych metod celem oceny
        wiarygodności ofert
        \begin{enumerate}[noitemsep,topsep=1pt]
            \item KMeansEvaluator - brak parametryzacji
            \item FuzzyCMeansEvaluator - brak parametryzacji
            \item BenchmarkEvaluator - parametry określone w nagłówku macierzy pomyłek
            \item BenchmarkEvaluator - parametry określone w nagłówku macierzy pomyłek
        \end{enumerate}
        \item W ramach każdego eksperymentu był mierzony czas wykonania oraz były
        uzyskiwane wyniki o~liczbie ofert wiarygodnych oraz niegodnych zaufania
        \item Uzyskane wyniki są porównywane z~wartościami ręcznej oceny
        wiarygodności i na tej podstawie jest tworzona macierz pomyłek.
    \end{enumerate}

    \section{Zbiór danych \#1}
    \label{chapter5:eksperymenty:zbior:1} {
        \subfile{experiment/dataset1.tex}
    }

    \section{Zbiór danych \#2}
    \label{chapter5:eksperymenty:zbior:2} {
        \subfile{experiment/dataset2.tex}
    }

    \section{Zbiór danych \#3}
    \label{chapter5:eksperymenty:zbior:3} {
        \subfile{experiment/dataset3.tex}
    }

    \section{Zbiór danych \#4}
    \label{chapter5:eksperymenty:zbior:4} {
        \subfile{experiment/dataset4.tex}
    }

    \section{Podsumowanie eksperymentów}
    \label{chapter5:eksperymenty:podsumowanie} {
        Po przeanalizowaniu wyników uzyskanych w~ramach eksperymentów dla wszystkich
        zbiorów można zauważyć tendencje, że opracowana metoda wykazuję się lepszą
        skutecznością niż ta zaimplementowana na podstawie literatury. Autor określił
        to po obserwacji macierzy pomyłek, metoda z~literatury popełniała więcej błędów
        pierwszego i~drugiego rodzaju. Oczywiście zaproponowana metoda również
        popełniała błędy i~w~niektórych przypadkach była ich niemała liczba. Nie mniej
        i~tak było ich zauważalnie mniej niż w~przypadku metody literaturowej. Warto
        dodać, że wartości określone przez człowieka są w~większości przypadków
        poprawne, lecz człowiek też może podejmować różne decyzje stąd mogą występować
        rozbieżności. Ocenę człowieka nigdy nie można traktować jako prawdę absolutną
        lecz informacje te są niezwykle przydatne przy badaniu takich algorytmów
        i~właśnie z~nimi można porównywać uzyskane wyniki.

        W~przypadku metody literaturowej były przeprowadzane badania dla różnych
        wartości parametrów progów, które zostały opisane w~sekcji
        \ref{chapter4:srodowisko_eksperymentalne:impl_literaturowej_metody}, lecz do
        niniejszej pracy magisterskiej autor wybierał dwa najlepsze wyniki dla parametrów.

        Po analizie autorskiej metody można zauważyć ciekawą kwestię, że dokonuje ona
        takich samych decyzji i~popełnia takie same błędy. Warto jednak spojrzeć na
        różnice w~czasach wykonania. Są one dosyć spore, w~niektórych przypadkach nawet
        kilkukrotnie mniejsze, oczywiście na korzyść wersji wykorzystującej C-Means.
        Może wynikach to z~różnic w~implementacji wykorzystanych bibliotek i~modułów.
        Jedynym zadziwiającym przypadkiem są wyniki wykonania algorytmów dla zbioru
        danych \#2 przedstawionego w~sekcji \ref{chapter5:eksperymenty:zbior:2}, może
        to wynikać z~kwestii sprzętowych lub jakieś ukrytej optymalizacji interpretera
        języka Python, na którą autor pracy nie ma wpływu.

        W~przypadku zbiorów danych o~niewielkiej liczbie rekordów tak jak w~przypadku
        przeprowadzanych eksperymentów, nie ma to praktycznie znaczenia w~realnym
        użyciu. Duża różnica mogłaby być, gdyby zbiory miały, chociażby jeden milion
        rekordów. Nie mniej taka sytuacji raczej nie miałaby miejsca, gdyż na portalach
        ogłoszeniowych konkretny produkt katalogowy raczej nie występuję w~tylu ofertach.
    }

}
\end{document}
